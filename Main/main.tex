\documentclass{article}
\usepackage[utf8]{inputenc}
\usepackage{amsmath}
\title{Lista-4}
\author{Homeque Vieira Martins - 642725}
\date{May 2021}
\begin{document}

\maketitle
\section{Questão 01}

    \begin{flushleft}
    
    $S_x$ = 30
    
    $S_y$ = 11240
    
    $S_x^2$ = 128
    
    $S_y^2$ = 20353600
    
    $S_{xy}$ = 50480

    \end{flushleft}
    
        \subsection{Item A}

            \begin{flushleft}

            $S_{xx}$= $n\Sigma x^2$ - $(\Sigma x)^2$ 
            
            $S_{xx}$=8*128-900

            $S_{xx}$=1024-900
            
            $S_{xx}$=124

            \end{flushleft}
        
            \begin{flushleft}
                $S_{yy}$= $n\Sigma y^2$ - $(\Sigma y)^2$

                $S_{yy}$=8*20353600-126337600

                $S_{yy}$=162828800-126337600

                $S_{yy}$=36491200

            \end{flushleft}

            \begin{flushleft}

               $S_{xy}$=$n\Sigma xy-(\Sigma x)(\Sigma y)$
               
               $S_{xy}$=8*50480-337200
               
               $S_{xy}$=403840-337200 
               
               $S_{xy}$=66640 
    
            \end{flushleft}

            \begin{flushleft}
                $Corr(x,y) = \frac{Sxy}{\sqrt{Sxx * Syy}} = \frac{66640}{\sqrt{124 * 36491200}} = \frac{66640}{67267} = 0,9907 $

            \end{flushleft}
            
            Resultado: Forte correlação positiva
    \subsection{Item B}
    
            \begin{flushleft}

                $\hat{\beta} = \frac{Sxy}{Sxx} = \frac{66640}{124} = 537,42$

                $\hat{\beta} = \bar{y} - \hat{\beta} * \bar{x} = 1405 -537,42 * 3,75 = -610,325$
            \end{flushleft}
            
            \subsubsection{Resultados e Analises}

            $\hat{\beta}0 $: Não possui um análise pratica \\    
            $\hat{\beta}1 $: Vocabulário medio de cada criança vem aumentando 537,42 palavras a cada ano

    \subsection{Item C}
            
            \begin{flushleft}
                $R^2 = (0,9907)^2$ =$0.9815 $  \ ou\  $ 98.15\$$
            \end{flushleft}

        \subsubsection{Resultado}
        Podemos entende que $98,15\% $ da palavras no vocabulário de uma criança depende da idade, 
        já os outros $1,85\%$, muito provalvemente vem de erros ou outras variável não aborada no estudo
    
    \subsection{Item D}
    %Formula 
    \begin{flushleft}
        $Se = \sqrt{ \frac{\Sigma y^2 - \hat{\beta } \Sigma y - \hat{\beta}1 \Sigma xy }{n-2}}$
    \end{flushleft}   


    \begin{flushleft}
        $Se = \sqrt{ \frac{20353600 - (610,325 * 11240)+ (537,42 * 50480) }{6}}$

        $Se = \sqrt{ \frac{84691,4 }{6}}$

        $Se = 118,81$
    \end{flushleft}   

    \begin{flushleft}
        $\hat{y} = \hat{B}0 + \hat{B}1 *n = -610,325 + 537,42 * 7=3151,615$
    \end{flushleft}   

    \begin{flushleft}
        $Ic(95\%) = \hat{y} + t\frac{a}{2};n-2 *Se* \sqrt{1 + \frac{1}{n} +\frac{N(x_0 -x)^2}{}}$

        $Ic(95\%) = 3151,615 \pm 2,4469 *  (118,81* \sqrt{1 + \frac{1}{8} + \frac{8(7-3,75)^2}{124}})$
        
        $Ic(95\%) =  3151,615 \pm 390,73$

        $Ic(95\%) = [2760,885$ \ ; \ $3542,345]$
    \end{flushleft}


    \subsection{Item E}

    \begin{flushleft}
        $IC(95\%) = \hat{y} \pm  t \frac{a}{2}; n-2 *Se * \sqrt{\frac{1}{n} + \frac{n(n_0 - x)}{Sxx}}$ 
        
        $Ic(95\%) = 3151,615 \pm 2,4469 *  (118,81* \sqrt{\frac{1}{8} + \frac{8(7-3,75)^2}{124}})$
        
        $Ic(95\%) =  3151,615 \pm 261,07$

        $Ic(95\%) = [2890,545$ \ ; \ $3412,685]$


    \end{flushleft}


\section{Questão 02}   
        \begin{flushleft}
            
            $Sx = 60$ 	

            $Sy = 891$
            
            $S^2 = 346$
            
            $Sy^2 = 65451$
            
            $Sxy = 4620$
        \end{flushleft}


    \subsection*{Item A}
    
    
            \begin{flushleft}

            $S_{xx} = n\Sigma x^2$ - $(\Sigma x)^2$ 
            
            $S_{xx} = 13*346-(60)^2$

            $S_{xx} = 4498-3600$
            
            $S_{xx}= 898$

            \end{flushleft}

            \begin{flushleft}
                $S_{yy}$= $n\Sigma y^2$ - $(\Sigma y)^2$

                $S_{yy} = 13*65451-(891)^2$

                $S_{yy} = 850863-793881$

                $S_{yy} = 56982$

            \end{flushleft}

            \begin{flushleft}

            $S_{xy} = n\Sigma xy-(\Sigma x)(\Sigma y)$
            
            $S_{xy} = 13*4620-(60)(891)$
            
            $S_{xy} = 60060-53460$ 
            
            $S_{xy} = 6600$ 

            \end{flushleft}

            \begin{flushleft}
                $Corr(x,y) = \frac{Sxy}{\sqrt{Sxx * Syy}} = \frac{6600}{\sqrt{898 * 56982}} =
                 \frac{6600}{51169836} =\frac{6600}{7153,30} = 0,9226 $
            \end{flushleft}
    
            Resultado: Fonte correlação positiva

    \subsection{Item B}

            \begin{flushleft}

                $\hat{\beta} = \frac{Sxy}{Sxx} = \frac{66640}{898} = 7,35$

                $\hat{\beta} = \bar{y} - \hat{\beta} * \bar{x} = 68,54-7,35* 4,62=34,58$
            \end{flushleft}
            
            \subsubsection{Resultados e Analises}

            $\hat{\beta}0 $: Ao alunos que possuiem zero hora de estudo
                             a media de nota esperada é 34,58 pontos   \\    
            $\hat{\beta}1 $: A cada hora de estudo adicional, a media de pontos do aluno aumenta 7,35

    \subsection{Item C}
        
            \begin{flushleft}
                $R^2 = (0,9907)^2$ =$0,8512$ \ ou \ $85,12\$$
            \end{flushleft}

            \subsubsection{Resultado}
            \paragraph{}Podemos atribuir 85,12\% da variabilidade na pontuação do teste as horas de estudo, 
            já os outros 14,88\% são 

    \subsection{Item D}
    %Formula 
            \begin{flushleft}
            $Se = \sqrt{ \frac{\Sigma y^2 - \hat{\beta } \Sigma y - \hat{\beta}1 \Sigma xy }{n-2}}$
            \end{flushleft}   

            \begin{flushleft}
            $Se = \sqrt{ \frac{65451 - (34,58 * 891)+ (7,35 * 4620) }{11}}$

            $Se = \sqrt{ \frac{683,22}{11}}$

            $Se = 7,88$
            \end{flushleft}   

            \begin{flushleft}
            $\hat{y} = \hat{B}0 + \hat{B}1 *n = 34,58+7,35 * 3=56,63$
            \end{flushleft}   

            \begin{flushleft}
            $Ic(95\%) = \hat{y} + t\frac{a}{2};n-2 *Se* \sqrt{1 + \frac{1}{n} +\frac{N(x_0 -x)^2}{}}$

            $Ic(95\%) = 56,63 \pm 3,1058 *  (7,88* \sqrt{1 + \frac{1}{13} + \frac{13(3-4,62)^2}{898}})$

            $Ic(95\%) =  56,63 \pm 25,84$

            $Ic(95\%) = [30,79$ \ ; \ $82,47]$
    \end{flushleft}


    \subsection{Item E}

            \begin{flushleft}
            $IC(95\%) = \hat{y} \pm  t \frac{a}{2}; n-2 *Se * \sqrt{\frac{1}{n} + \frac{n(n_0 - x)}{Sxx}}$ 

            $Ic(95\%) = 56,63 \pm 3,1058 *  (7,88* \sqrt{\frac{1}{13} + \frac{13(3-4,62)^2}{898}})$

            $Ic(95\%) =  56,63 \pm 8,29$

            $Ic(95\%) = [48,34$ \ ; \ $64,92]$

            \end{flushleft}


\section{Questão 03}    

    \subsection*{Item A}

            $S_x$ = 54
            
            $S_y$ = 908
            
            $S_x^2$ = 332
            
            $S_y^2$ = 70836
            
            $S_{xy}$ = 3724

            \begin{flushleft}

            $S_{xx}$= $n\Sigma x^2$ - $(\Sigma x)^2$ 

            $S_{xx} =12*332 -(54)^2$

            $S_{xx} = 3984 -2916 $

            $S_{xx}= 168$

            \end{flushleft}

            \begin{flushleft}
                $S_{yy}$= $n\Sigma y^2$ - $(\Sigma y)^2$

                $S_{yy} = 12* 70836 - (908)^2$

                $S_{yy} = 850032 - 824464 $

                $S_{yy} = 25568$

            \end{flushleft}

            \begin{flushleft}

            $S_{xy} = n\Sigma xy-(\Sigma x)(\Sigma y)$

            $S_{xy} = 12*3724-(54)(908)$

            $S_{xy} = 44688-49032$ 

            $S_{xy} = -4344$ 

            \end{flushleft}

            \begin{flushleft}
                $Corr(x,y) = \frac{Sxy}{\sqrt{Sxx * Syy}} = \frac{-4344}{\sqrt{1068 * 25568}} =
                            \frac{-4344}{\sqrt{27306624}} = \frac{-4344}{5225,57} = -0,8313 $

            \end{flushleft}

            Resultado: Forte correlação negativa

    \subsection{Item B}

            \begin{flushleft}

                $\hat{\beta} = \frac{Sxy}{Sxx} = \frac{-4344}{1068} = -4,07$

                $\hat{\beta} = \bar{y} - \hat{\beta} * \bar{x} = 75.66+4.07* 4.5 = 93.97$
            \end{flushleft}

            \subsubsection{Resultados e Analises}

            $\hat{\beta}0 $: É esperado que um aluno
            que assistem zero horas de televisão, uma nota média de 93,97 pontos  \\    
            $\hat{\beta}1 $: Para cada hora adicional para assistir televisão,
                    temos uma diminuição na média de -4,07 pontos na nota. 

    \subsection{Item C}

            \begin{flushleft}
                $R^2 = (-0,8313)^2$ =$0,691$ ou $69,10\$$
            \end{flushleft}

            \subsubsection{Resultado}
            \paragraph{}A pontuação de $69,10\% $ é explicado pelas horas de televisão assistindas, 
                        já os outros $30,90\%$ não são explicadas
                        pois o estudado não abrange ou houve erros 

    \subsection{Item D}
            %Formula 
            \begin{flushleft}
            $Se = \sqrt{ \frac{\Sigma y^2 - \hat{\beta } \Sigma y - \hat{\beta}1 \Sigma xy }{n-2}}$
            \end{flushleft}   

            \begin{flushleft}
            $Se = \sqrt{ \frac{70836 - (93,97 * 908)+ (4,07 * 3724) }{10}}$

            $Se = \sqrt{ \frac{667,92}{10}}$

            $Se = 8,17$
            \end{flushleft}   

            \begin{flushleft}
            $\hat{y} = \hat{B}0 + \hat{B}1 *n = -610,325 + 537,42 * 7=3151,615$
            \end{flushleft}   

            \begin{flushleft}
            $Ic(95\%) = \hat{y} + t\frac{a}{2};n-2 *Se* \sqrt{1 + \frac{1}{n} +\frac{N(x_0 -x)^2}{}}$

            $Ic(95\%) = 57,34 \pm 2,2281 *  (8,17* \sqrt{1 + \frac{1}{12} + \frac{12(9-4,5)^2}{1068}})$

            $Ic(95\%) =  57,34 \pm 20,84$

            $Ic(95\%) = [36,5$ \ ; \ $78,18]$
            \end{flushleft}


\subsection{Item E}

\begin{flushleft}
$IC(95\%) = \hat{y} \pm  t \frac{a}{2}; n-2 *Se * \sqrt{\frac{1}{n} + \frac{n(n_0 - x)}{Sxx}}$ 

$Ic(95\%) = 57,34 \pm 2,2281 *  (8,17* \sqrt{\frac{1}{12} + \frac{12(9-4,5)^2}{1068}})$

$Ic(95\%) =  57,34 \pm 10,15$

$Ic(95\%) = [47,19$ \ ; \ $67,49]$

\end{flushleft}



\end{document}
